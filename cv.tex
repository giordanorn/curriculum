\documentclass[9pt]{developercv}


\firstname{Giordano}
\lastname{Vicente}

\email{\href{mailto:xxxxxxx@xxxxx.xxx}{xxxxxxx@xxxxx.xxx}}
\phone{+55 (xx) xxxxx-xxxx}
\location{xxxxxxx -- xx, xxxxxx. \\xx. xxxxx CEP: xxxxx-xxx}

\website{\href{https://giordanorn.github.io}{giordanorn.github.io}}
\github{\href{https://github.com/giordanorn}{giordanorn}}
\linkedin{\href{https://www.linkedin.com/in/giordano-vicente/}{giordano-vicente}}


\begin{document}
	\maketitle

	\cvsect{Sobre mim}
	\begin{minipage}[t]{1\textwidth}
		Me chamo Giordano, atualmente estou gradu\-ando no Bacharelado em Tecnologia da Informação (BTI) pela Universidade Federal do Rio Grande do Norte (UFRN).
		Adoro interagir com as pessoas, passar e receber conhecimentos. Sou aspirante a cientista da computação, matemático, filósofo e músico. Entusiasta de hacking e sistemas UNIX. 
	\end{minipage}
	
	\cvsect{Habilidades Técnicas}
	\begin{minipage}[t]{1\textwidth}
		Possuo experiência profissional em Desenvolvimento Web Front-End com HTML5, JavaScript (Vue.js) e CSS (SASS). Além de Git e UNIX. Programação em C e Haskell. Bastante experiência em \TeX{} e \LaTeX{}.
	\end{minipage}


	\cvsect{Atuação Profissional}
	\begin{entrylist}
		\entry
			{8/2020 -- atual}
			{Desenvolvimento Web Front-End\\
				\footnotesize{Estágio}}
            {Ministério Público do Rio Grande do Norte (MPRN)}
			{Responsável pelo desenvolvimento de sistemas relacionados ao MPRN.
                Competências e tecnologias envolvidas: HTML5 (Pug), JavaScript (Vue.js, Nuxt.js), Testes de integração (Cypress), CSS (SASS), Git e SCRUM.}
		\entry
			{1/2018 -- 8/2018}
			{Desenvolvimento Web Front-End\\
				\footnotesize{Estágio}}
			{\href{https://www.info.ufrn.br/}{Superintendência de Informática da UFRN (SINFO)}}
			{Responsável pelo desenvolvimento e manutenção de sites relacionados à UFRN.
				Competências e tecnologias envolvidas: HTML5 (Pug), JavaScript (Vue.js), CSS (SASS), Git e SCRUM.}
		\entry
			{12/2016 -- 3/2017}
			{Suporte de Informática\\
				\footnotesize{Estágio}}
			{\href{http://unirn.edu.br/}{Centro Universitário do Rio Grande do Norte (UNI-RN)}}
			{Responsável pela manutenção e suporte de sistemas operacionais dos laboratórios de estudo da UNI-RN.
				Competências e tecnologias envolvidas: FOG e Linux.}
	\end{entrylist}


	\cvsect{Atuação Acadêmica}
	\begin{entrylist}
		\entry
			{12/2019}
			{Oficina de Introdução ao \LaTeX{}\\
				\footnotesize{{Projeto de Extensão}}}
			{(UFRN)}
			{Ofertatada durante o evento ``III WGIIfE - Workshop do Grupo Interdisciplinar de estudos e pesquisas em Informática na Educação'' como uma oficina aberta aos participantes.}
		\entry
			{3/2019 -- 12/2019}
			{Monitoria de Matemática Elementar\\
				\footnotesize{Projeto de Ensino}}
			{(UFRN)}
			{Orientado pelo Prof. MSc. Antonio Igor Silva de Oliveira. Monitor das disciplinas:\\
				\texttt{IMD1001 - MATEMÁTICA ELEMENTAR}}
		\entry
			{1/2019 -- atual}
			{Organização da produção colaborativa do material textual escrito em \LaTeX{}.\\
				\footnotesize{Projeto de Pesquisa}}
			{(UFRN)}
			{Orientado pelo Prof. MSc. Antonio Igor Silva de Oliveira. Plano de trabalho do projeto intitulado ``Uso da Khan Academy como prática auxiliar de ensino da disciplina Matemática Elementar do IMD''.}
		\entry
			{8/2018 -- 12/2018}
			{Projeto de monitoria FMCn\\
				\footnotesize{Projeto de Ensino}}
			{(UFRN)}
			{Orientado pelo Prof. Dr. Athanasios Tsouanas. Monitor das disciplinas:\\
				\texttt{IMD0028 - FUNDAMENTOS MATEMÁTICOS DA COMPUTAÇÃO I};\\
				\texttt{IMD0038 - FUNDAMENTOS MATEMÁTICOS DA COMPUTAÇÃO II}.}
		\entry
			{7/2018}
			{Minicurso de Introdução à Lógica Clássica Proposicional e Técnicas de Demonstração Matemática\\
				\footnotesize{{Projeto de Extensão}}}
			{(UFRN)}
			{}
		\entry
			{4/2018 -- 12/2018}
			{Monitoria de Matemática Elementar\\
				\footnotesize{Projeto de Ensino}}
			{(UFRN)}
			{Orientado pelo Prof. MSc. Antonio Igor Silva de Oliveira. Monitor das disciplinas:\\
				\texttt{IMD1001 - MATEMÁTICA ELEMENTAR}.}
	\end{entrylist}


	\cvsect{Formação Acadêmica}
	\begin{entrylist}
		\entry
			{2015 - atual}
			{Bacharelado em Tecnologia da Informação\\
				\footnotesize{Graduação}}
			{(UFRN)}
			{}
	\end{entrylist}


	\cvsect{Idiomas}
	\begin{minipage}[t]{1\textwidth}
		\textbf{Português} - Nativo;\\
		\textbf{Inglês} - Falo e compreendo bem.
	\end{minipage}
\end{document}
