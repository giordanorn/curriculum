%%%%%%%%%%%%%%%%%%%%%%%%%%%%%%%%%%%%%%%%%
% Developer CV
% LaTeX Template
% Version 1.0 (28/1/19)
%
% This template originates from:
% http://www.LaTeXTemplates.com
%
% Authors:
% Jan Vorisek (jan@vorisek.me)
% Based on a template by Jan Küster (info@jankuester.com)
% Modified for LaTeX Templates by Vel (vel@LaTeXTemplates.com)
%
% License:
% The MIT License (see included LICENSE file)
%
%%%%%%%%%%%%%%%%%%%%%%%%%%%%%%%%%%%%%%%%%

%----------------------------------------------------------------------------------------
%	PACKAGES AND OTHER DOCUMENT CONFIGURATIONS
%----------------------------------------------------------------------------------------

\documentclass[9pt]{developercv} % Default font size, values from 8-12pt are recommended

%----------------------------------------------------------------------------------------

\begin{document}

%----------------------------------------------------------------------------------------
%	TITLE AND CONTACT INFORMATION
%----------------------------------------------------------------------------------------

\begin{minipage}[t]{0.45\textwidth} % 45% of the page width for name
    \vspace{-\baselineskip} % Required for vertically aligning minipages
    
    % If your name is very short, use just one of the lines below
    % If your name is very long, reduce the font size or make the minipage wider and reduce the others proportionately
    \colorbox{black}{{\HUGE\textcolor{white}{\textbf{\MakeUppercase{Giordano}}}}} % First name
    
    \hspace{30pt} \colorbox{black}{{\HUGE\textcolor{white}{\textbf{\MakeUppercase{Vicente}}}}} % Last name
    
    \vspace{6pt}
    
    % {\huge Web App Architect} % Career or current job title
\end{minipage}
\begin{minipage}[t]{0.275\textwidth} % 27.5% of the page width for the first row of icons
    \vspace{-\baselineskip} % Required for vertically aligning minipages
    
    % The first parameter is the FontAwesome icon name, the second is the box size and the third is the text
    % Other icons can be found by referring to fontawesome.pdf (supplied with the template) and using the word after \fa in the command for the icon you want
    \icon{MapMarker}{12}{Natal - RN}\\
    %\icon{Phone}{12}{+55 (xx) xxxxx-xxxx}\\
    \icon{At}{12}{\href{mailto:giordanovofr@gmail.com}{giordanovofr@gmail.com}}\\	
    \icon{Institution}{12}{\href{https://ufrn.br}{IMD -- UFRN}}\\
\end{minipage}
\begin{minipage}[t]{0.275\textwidth} % 27.5% of the page width for the second row of icons
    \vspace{-\baselineskip} % Required for vertically aligning minipages
    
    % The first parameter is the FontAwesome icon name, the second is the box size and the third is the text
    % Other icons can be found by referring to fontawesome.pdf (supplied with the template) and using the word after \fa in the command for the icon you want
    \icon{Globe}{12}{\href{https://giordanorn.github.io}{giordanorn.github.io}}\\
    \icon{Github}{12}{\href{https://github.com/giordanorn}{github.com/giordanorn}}\\
    \icon{Linkedin}{12}{\href{https://www.linkedin.com/in/giordano-vicente-577087129/}{giordano-vicente}}\\
\end{minipage}

\vspace{0.5cm}

%----------------------------------------------------------------------------------------
%	INTRODUCTION, SKILLS AND TECHNOLOGIES
%----------------------------------------------------------------------------------------

\cvsect{Sobre mim}

\begin{minipage}[t]{0.4\textwidth} % 40% of the page width for the introduction text
    \vspace{-\baselineskip} % Required for vertically aligning minipages
    
    Me chamo Giordano, atualmente estou graduando no Bacharelado em Tecnologia da Informação (BTI) pela Universidade Federal do Rio Grande do Norte (UFRN).
    
    Adoro interagir com as pessoas, passar e receber conhecimentos. Também adoro hackear e programar. Sou aspirante de cientista da computação, matemático, filósofo e músico.\\
\end{minipage}
\hfill % Whitespace between
\begin{minipage}[t]{0.5\textwidth} % 50% of the page for the skills bar chart
    \vspace{-\baselineskip} % Required for vertically aligning minipages
    \cvsect{Habilidades Técnicas}
    \begin{barchart}{5.5}
        \baritem{JavaScript}{55}
        \baritem{SASS}{50}
        \baritem{Git}{80}
        \baritem{Linux}{70}
        \baritem{C}{60}
        \baritem{Haskell}{30}
        \baritem{LaTeX}{80}
    \end{barchart}
\end{minipage}

%\begin{center}
%    \bubbles{5/a, 6/b, 4/c, 3/d, 1/e}
%\end{center}

%----------------------------------------------------------------------------------------
%	EXPERIENCE
%----------------------------------------------------------------------------------------

\cvsect{Experiência Profissional}

\begin{entrylist}
    \entry
        {1/2018 -- 8/2018}
        {Desenvolvimento Web Front-End\\
        \footnotesize{Estágio}}
        {Superintendência de Informática da UFRN (SINFO)}
        {\texttt{HTML5}\slashsep\texttt{Vue.js}\slashsep\texttt{SASS}\slashsep\texttt{Git}\slashsep\texttt{SCRUM}}
    \entry
        {12/2016 -- 3/2017}
        {Suporte de Informática\\
        \footnotesize{Estágio}}
        {Centro Universitário do Rio Grande do Norte (UNI-RN)}
        {\texttt{FOG}\slashsep\texttt{Linux}}
\end{entrylist}

%----------------------------------------------------------------------------------------
%	EXPERIENCE
%----------------------------------------------------------------------------------------

\cvsect{Experiência Acadêmica}

\begin{entrylist}
    \entry
        {12/2019}
        {Minicurso de Introdução ao \LaTeX{}
        \\\footnotesize{{Projeto de Extensão}}}
        {UFRN}
        {Ofertatado durante o III WGIIfE - Workshop do Grupo Interdisciplinar de estudos e pesquisas em Informática na Educação como uma oficina aberta aos participantes do evento.}
    \entry
        {3/2019 -- 12/2019}
        {Monitoria de Matemática Elementar\\
        \footnotesize{Projeto de Ensino}}
        {UFRN}
        {Orientado pelo Prof. MSc. Antonio Igor Silva de Oliveira. Monitor das disciplinas:\\
        \texttt{IMD1001 - MATEMÁTICA ELEMENTAR}}
    \entry
        {8/2018 -- 12/2018}
        {Projeto de monitoria FMCn\\
        \footnotesize{Projeto de Ensino}}
        {UFRN}
        {Orientado pelo Prof. Dr. Athanasios Tsouanas. Monitor das disciplinas:
        \\\texttt{IMD0028 - FUNDAMENTOS MATEMÁTICOS DA COMPUTAÇÃO I};
        \\\texttt{IMD0038 - FUNDAMENTOS MATEMÁTICOS DA COMPUTAÇÃO II}.}
    \entry
        {07/2018}
        {Minicurso de Introdução à Lógica Clássica Proposicional e Técnicas de Demonstração Matemática\\
        \footnotesize{{Projeto de Extensão}}}
        {UFRN}
        {}
    \entry
        {4/2018 -- 12/2018}
        {Monitoria de Matemática Elementar\\
        \footnotesize{Projeto de Ensino}}
        {UFRN}
        {Orientado pelo Prof. MSc. Antonio Igor Silva de Oliveira. Monitor das disciplinas:\\
        \texttt{IMD1001 - MATEMÁTICA ELEMENTAR}.}
\end{entrylist}

%----------------------------------------------------------------------------------------
%	EDUCATION
%----------------------------------------------------------------------------------------

\cvsect{Educação}

\begin{entrylist}
    \entry
        {2015 -- atual}
        {Bacharelado em Tecnologia da Informação\\\footnotesize{Graduação}}
        {UFRN}
        {Ingresso no semestre 2015.2.}
\end{entrylist}

%----------------------------------------------------------------------------------------
%	ADDITIONAL INFORMATION
%----------------------------------------------------------------------------------------

\begin{minipage}[t]{0.3\textwidth}
    \vspace{-\baselineskip} % Required for vertically aligning minipages

    \cvsect{Idiomas}
    
    \textbf{Português} - nativo\\
    \textbf{Inglês} - razoável\\
\end{minipage}
% \hfill
% \begin{minipage}[t]{0.3\textwidth}
%     \vspace{-\baselineskip} % Required for vertically aligning minipages
    
%     \cvsect{Hobbies}
    
%     I love... \lorem
% \end{minipage}
% \hfill
% \begin{minipage}[t]{0.3\textwidth}
%     \vspace{-\baselineskip} % Required for vertically aligning minipages
    
%     \cvsect{Non profit}
    
%     I help... \lorem
% \end{minipage}

%----------------------------------------------------------------------------------------

\end{document}
